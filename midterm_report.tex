\documentclass[11pt,letterpaper]{article}

% Packages
\usepackage[utf8]{inputenc}
\usepackage[margin=1in]{geometry}
\usepackage{amsmath,amssymb,amsthm}
\usepackage{graphicx}
\usepackage{hyperref}
\usepackage{listings}
\usepackage{xcolor}
\usepackage{booktabs}
\usepackage{float}
\usepackage{caption}
\usepackage{subcaption}
\usepackage{algorithm}
\usepackage{algpseudocode}
\usepackage{enumitem}
\usepackage{fancyhdr}
\usepackage{titlesec}

% Hyperref setup
\hypersetup{
    colorlinks=true,
    linkcolor=blue,
    filecolor=magenta,      
    urlcolor=cyan,
    citecolor=blue,
    pdftitle={Traffic Flow Optimization - Midterm Report},
    pdfauthor={},
}

% Code listing setup
\lstset{
    basicstyle=\ttfamily\small,
    keywordstyle=\color{blue},
    commentstyle=\color{green!60!black},
    stringstyle=\color{red},
    showstringspaces=false,
    breaklines=true,
    frame=single,
    numbers=left,
    numberstyle=\tiny\color{gray},
    captionpos=b,
}

\lstdefinestyle{python}{
    language=Python,
    basicstyle=\ttfamily\footnotesize,
    keywordstyle=\color{blue},
    commentstyle=\color{green!60!black},
    stringstyle=\color{orange},
}

\lstdefinestyle{bash}{
    language=bash,
    basicstyle=\ttfamily\footnotesize,
    keywordstyle=\color{blue},
    commentstyle=\color{green!60!black},
}

\lstdefinestyle{yaml}{
    basicstyle=\ttfamily\footnotesize,
    keywordstyle=\color{blue},
    commentstyle=\color{green!60!black},
}

% Headers and footers
\pagestyle{fancy}
\fancyhf{}
\rhead{Traffic Flow Optimization}
\lhead{Midterm Report}
\rfoot{Page \thepage}

% Title formatting
\titleformat{\section}{\Large\bfseries}{\thesection}{1em}{}
\titleformat{\subsection}{\large\bfseries}{\thesubsection}{1em}{}
\titleformat{\subsubsection}{\normalsize\bfseries}{\thesubsubsection}{1em}{}

% Document metadata
\title{\textbf{Comparative Analysis of Street Intersections and Optimization Strategies}\\[0.5em]
\Large Midterm Report}
\author{Traffic Flow Optimization through Multi-Platform Microsimulation}
\date{October 19, 2025}

\begin{document}

\maketitle
\thispagestyle{empty}

\begin{abstract}
This project develops a dual-platform microsimulation framework to quantify how geometric design and behavioral parameters affect operational efficiency at traffic intersections, with initial focus on four-arm roundabouts. The framework consists of (1) a text-based Python microsimulation implementing continuous car-following dynamics via the Intelligent Driver Model (IDM), delay-differential equations (DDEs) for reaction time, and stochastic gap acceptance, and (2) a SUMO-based validation pipeline that replicates the Python simulation's modeling logic for cross-platform validation. The absence of graphics in the Python simulator enables rapid parameter sweeps and reproducible measurements of throughput, delay, and queue dynamics. SUMO augments this analysis with visual diagnostics, emissions modeling, and established validation against real-world data. Future phases will extend this framework to signalized intersections with reinforcement learning for adaptive signal control.
\end{abstract}

\newpage
\tableofcontents
\newpage

\section{Introduction}

This project develops a dual-platform microsimulation framework to quantify how geometric design and behavioral parameters affect operational efficiency at traffic intersections, with initial focus on four-arm roundabouts. The framework consists of:

\subsection{Text-Based Python Microsimulation}
A single-file, configurable simulator (\texttt{Roundabout.py}) implementing continuous car-following dynamics via the Intelligent Driver Model (IDM), delay-differential equations (DDEs) for reaction time, stochastic gap acceptance, and Poisson arrival processes. The absence of graphics enables rapid parameter sweeps and reproducible measurements of throughput, delay, and queue dynamics.

\subsection{SUMO-Based Validation Pipeline}
A complete production pipeline using SUMO (Simulation of Urban Mobility) that replicates the Python simulation's modeling logic—comparable queuing rules at entries, consistent car-following dynamics, and aligned demand patterns—enabling cross-platform validation and leveraging SUMO's visual diagnostics, emissions modeling, and ecosystem for scenario management.

\subsection{Project Objectives}

\begin{itemize}[leftmargin=*]
    \item \textbf{Capacity Analysis}: Identify breaking points where geometric modifications (e.g., additional lanes, larger diameter) become necessary
    \item \textbf{Design Optimization}: Determine optimal configurations across multiple objectives (throughput, delay, emissions)
    \item \textbf{Comparative Assessment}: Establish which intersection control strategy (roundabout vs. signalized) performs better under various demand scenarios
    \item \textbf{Model Validation}: Use cross-platform agreement as evidence for model correctness; investigate discrepancies to refine assumptions
\end{itemize}

\subsection{Dual-Platform Rationale}

Developing our own Python simulation provides complete transparency over assumptions, algorithms, and metrics, facilitating deep understanding of how modeling choices influence outcomes. SUMO augments this analysis with:

\begin{itemize}[leftmargin=*]
    \item Visual flow inspection and debugging
    \item Richer built-in diagnostics (emissions, fuel consumption, noise)
    \item Established validation against real-world data
    \item Integration with traffic assignment and demand modeling tools
\end{itemize}

Future phases will extend this framework to signalized intersections (Phase 2) with reinforcement learning (RL) for adaptive signal control, and real-world application using OpenStreetMap data (Phase 3).

\section{Problem Statement \& Challenges}

\subsection{Core Problem}

Urban traffic intersections operate as critical bottlenecks in transportation networks. Poor design or control strategies lead to:

\begin{itemize}[leftmargin=*]
    \item \textbf{Capacity saturation}: Queue divergence when demand exceeds service capacity
    \item \textbf{Excessive delays}: Reduced level-of-service affecting user satisfaction
    \item \textbf{Safety concerns}: Increased conflict points and crash risk
    \item \textbf{Environmental impact}: Elevated emissions from idling and stop-and-go behavior
\end{itemize}

\textbf{Fundamental question}: Given demand characteristics (volume, turning movements) and site constraints, what intersection control strategy and geometric configuration maximize efficiency while ensuring stability?

\subsection{Technical Challenges}

\subsubsection{Stochastic Variability}

Traffic is inherently random:
\begin{itemize}[leftmargin=*]
    \item \textbf{Arrival processes}: Poisson or more complex (platooning, signal influence)
    \item \textbf{Driver heterogeneity}: Critical gap acceptance varies by driver
    \item \textbf{Turning choices}: Probabilistic movement selection
    \item \textbf{Behavioral uncertainty}: Reaction times, desired speeds, aggressiveness
\end{itemize}

\textbf{Solution}: Monte Carlo simulation with sufficient replication to capture statistical properties.

\subsubsection{Multi-Objective Optimization}

No single ``best'' design exists; trade-offs include:
\begin{itemize}[leftmargin=*]
    \item Maximize throughput $\leftrightarrow$ minimize delay
    \item Reduce emissions $\leftrightarrow$ increase flow rate
    \item Ensure safety $\leftrightarrow$ optimize capacity
\end{itemize}

\textbf{Solution}: Multi-objective optimization identifying Pareto-optimal configurations; decision-makers select based on priorities.

\subsubsection{Continuous Dynamics with Discrete Events}

Vehicles follow continuous ODEs (car-following, lane-changing) while experiencing discrete events (arrivals, merges, exits).

\textbf{Solution}: Hybrid simulation architecture with time-stepping for continuous dynamics and event scheduling for discrete transitions.

\subsubsection{Reaction Delay (Non-Markovian Effects)}

Human reaction time $\tau \approx 1$--2s means acceleration at time $t$ depends on states at $t-\tau$, creating delay-differential equations (DDEs).

\textbf{Solution (Python)}: History buffer storing vehicle states; retrieve delayed snapshots for IDM calculation.

\textbf{Solution (SUMO)}: \texttt{actionStepLength} approximates delayed responses (vehicles update control every $\tau$ seconds).

\subsubsection{Gap Acceptance at Yield Points}

Roundabout entries require modeling:
\begin{itemize}[leftmargin=*]
    \item \textbf{Critical gap} $T_c$: first vehicle needs larger gap to merge (right-skewed distribution $\rightarrow$ lognormal)
    \item \textbf{Follow-up headway} $T_f$: subsequent vehicles follow tighter (normal distribution)
    \item \textbf{Platoon dynamics}: spoiled opportunity breaks platoon; next attempt reverts to $T_c$
\end{itemize}

\textbf{Solution}: Per-vehicle stochastic gap draws (Python) or threshold with impatience growth (SUMO).

\subsubsection{Cross-Platform Consistency}

Ensuring comparable results between Python and SUMO requires:
\begin{itemize}[leftmargin=*]
    \item Parameter mapping (e.g., lateral acceleration $\rightarrow$ ring speed limit)
    \item Equivalent arrival processes
    \item Matched car-following models
    \item Aligned failure criteria
\end{itemize}

\textbf{Solution}: Comprehensive parameter mapping document; validation protocol comparing distributions, not just means.

\section{Mathematical \& Computational Background}

\subsection{Poisson Arrival Process}

\textbf{Model}: Vehicle arrivals at each approach follow a homogeneous Poisson process with rate $\lambda$ (vehicles/second).

\subsubsection{Counts View}

The number of arrivals $N(T)$ in interval $[0,T]$ is Poisson-distributed:
\begin{equation}
P(N(T) = k) = \frac{(\lambda T)^k e^{-\lambda T}}{k!}
\end{equation}

\textbf{Properties}:
\begin{itemize}[leftmargin=*]
    \item Mean: $\mathbb{E}[N(T)] = \lambda T$
    \item Variance: $\text{Var}[N(T)] = \lambda T$ (variability matches mean)
\end{itemize}

\subsubsection{Gaps View (Interarrivals)}

Time between successive vehicles $\Delta t \sim \text{Exponential}(\lambda)$:
\begin{equation}
f(\Delta t) = \lambda e^{-\lambda \Delta t}, \quad \Delta t \geq 0
\end{equation}

\textbf{Memoryless property}: $P(\Delta t > t+s \mid \Delta t > t) = P(\Delta t > s)$

Average headway: $\mathbb{E}[\Delta t] = 1/\lambda$

\textbf{Implementation (Python)}:
\begin{lstlisting}[style=python]
def _next_arrival_time(self, arm: int, now: float) -> float:
    lam = max(1e-9, self.cfg.demand.arrivals[arm])   # λ (veh/s)
    return now + random.expovariate(lam)             # Δt ~ Exp(λ)
\end{lstlisting}

\textbf{Implementation (SUMO)}: Route files specify Poisson departures via exponentially distributed \texttt{depart} times.

\subsection{Turning Movement Selection}

Each vehicle selects Right/Through/Left with probabilities $(p_R, p_T, p_L)$ summing to 1.

\textbf{Categorical Sampling (Inverse-CDF Method)}:
\begin{enumerate}[leftmargin=*]
    \item Draw $U \sim \text{Uniform}[0,1)$
    \item If $U < p_R$: choose Right
    \item Else if $U < p_R + p_T$: choose Through
    \item Else: choose Left
\end{enumerate}

\textbf{Implementation (Python)}:
\begin{lstlisting}[style=python]
def _draw_turn_steps(self) -> int:
    L, T, R = self.cfg.demand.turning_LTR
    u = random.random()
    if u < R:       return 1    # Right
    elif u < R + T: return 2    # Through
    else:           return 3    # Left
\end{lstlisting}

\subsection{Gap Acceptance Model}

\subsubsection{Critical Gap (First Vehicle)}

\textbf{Lognormal Distribution} $T_c \sim \text{LogNormal}(\mu, \sigma^2)$:
\begin{equation}
f(t) = \frac{1}{t \sigma \sqrt{2\pi}} \exp\left(-\frac{(\ln t - \mu)^2}{2\sigma^2}\right), \quad t > 0
\end{equation}

\textbf{Parameter conversion} from mean $m$ and standard deviation $s$:
\begin{align}
\mu &= \ln\left(\frac{m^2}{\sqrt{s^2 + m^2}}\right) \\
\sigma^2 &= \ln\left(1 + \frac{s^2}{m^2}\right)
\end{align}

\textbf{Typical values}: $m \approx 3.0$s, $s \approx 0.8$s

\textbf{Rationale}: Lognormal produces positive, right-skewed gaps matching empirical data \cite{zheng2011}.

\subsubsection{Follow-Up Headway (Platooning)}

\textbf{Normal Distribution} $T_f \sim \mathcal{N}(\mu_f, \sigma_f^2)$:
\begin{equation}
f(t) = \frac{1}{\sigma_f \sqrt{2\pi}} \exp\left(-\frac{(t - \mu_f)^2}{2\sigma_f^2}\right)
\end{equation}

\textbf{Typical values}: $\mu_f \approx 2.0$s, $\sigma_f \approx 0.4$s (bounded below at 0.2s for safety)

\subsection{Intelligent Driver Model (IDM)}

\textbf{Continuous car-following model} computing acceleration based on:
\begin{itemize}[leftmargin=*]
    \item Own speed $v$
    \item Desired speed $v_0$
    \item Gap to leader $s$
    \item Speed difference $\Delta v = v - v_L$
\end{itemize}

\subsubsection{IDM Equation}

\begin{equation}
a = a_{\max} \left[1 - \left(\frac{v}{v_0}\right)^\delta - \left(\frac{s^*}{s}\right)^2\right]
\end{equation}

where \textbf{desired dynamical distance}:
\begin{equation}
s^* = s_0 + vT + \frac{v \Delta v}{2\sqrt{a_{\max} b}}
\end{equation}

\textbf{Parameters}:
\begin{itemize}[leftmargin=*]
    \item $s_0$: minimum gap (m) — typically 2.0m
    \item $T$: desired time headway (s) — typically 1.5s
    \item $a_{\max}$: maximum acceleration (m/s²) — typically 2.0 m/s²
    \item $b$: comfortable deceleration (m/s²) — typically 3.0 m/s²
    \item $\delta$: acceleration exponent — typically 4
\end{itemize}

\textbf{Properties}:
\begin{itemize}[leftmargin=*]
    \item Free-flow: When $s \to \infty$, $a \to a_{\max}[1-(v/v_0)^\delta]$ (accelerate toward $v_0$)
    \item Interaction: When $s \approx s^*$, second term activates smooth deceleration
    \item Emergency braking: If leader suddenly stops, $b$ term provides strong braking
\end{itemize}

\subsection{Delay-Differential Equations (DDEs)}

Human reaction time introduces delay: vehicle $i$ computes acceleration at time $t$ using states from $t-\tau$.

\subsubsection{Informal DDE Representation}

\begin{align}
\dot{v}_i(t) &= a_{\text{IDM}}(v_i(t-\tau), v_{i-1}(t-\tau), s_i(t-\tau)) \\
\dot{x}_i(t) &= v_i(t)
\end{align}

\textbf{Effect}: Damps unrealistic instantaneous reactions; produces smoother, more realistic platoon dynamics.

\textbf{Implementation (SUMO)}: \texttt{actionStepLength="1.0"} means vehicles update control every 1.0s, approximating $\tau=1.0$s delay (not a true DDE but functionally similar).

\subsection{Numerical Integration (Time-Stepping)}

Continuous ODEs are advanced using \textbf{forward Euler method} with time step $\Delta t$:
\begin{align}
v(t+\Delta t) &= v(t) + a(t) \cdot \Delta t \\
x(t+\Delta t) &= x(t) + v(t+\Delta t) \cdot \Delta t
\end{align}

\textbf{Speed clamping}: $v \in [0, v_0]$ for physical validity

\textbf{Typical $\Delta t$}: 0.1s (Python), 1.0s (SUMO's default simulation step)

\subsection{SUMO-Specific Mathematical Models}

\subsubsection{Lane-Changing Model (LC2013)}

SUMO's \texttt{laneChangeModel="LC2013"} implements:

\textbf{Strategic lane changes}: Multi-lane planning for route continuation (e.g., positioning for roundabout exit)

\textbf{Cooperative changes}: Yield to faster vehicles; assist merging traffic

\textbf{Incentive calculation}:
\begin{equation}
\Delta a = a_{\text{new lane}} - a_{\text{current lane}} + \text{bias}
\end{equation}

Change occurs if $\Delta a >$ threshold and safety gap satisfied.

\subsubsection{Webster's Method for Signal Timing Optimization}

For signalized intersections (4-way traffic signals), \textbf{F.V. Webster's method} provides optimal cycle lengths and phase splits to minimize average vehicle delay. This is foundational for Phase 2 traffic signal optimization.

\textbf{Context}: Unlike roundabouts (continuous gap-acceptance), signalized intersections operate on discrete \textbf{cycle-based control}:
\begin{itemize}[leftmargin=*]
    \item \textbf{Cycle time $C$}: Total time for one complete signal rotation (s)
    \item \textbf{Green time $g_i$}: Duration phase $i$ shows green (s)
    \item \textbf{Red time $r_i$}: Duration phase $i$ shows red (s)
    \item \textbf{Lost time $L$}: Time wasted during phase transitions (startup delay + amber + all-red)
\end{itemize}

\textbf{Critical Parameters}:
\begin{itemize}[leftmargin=*]
    \item \textbf{Flow rate $q_i$}: vehicles/hour on approach $i$
    \item \textbf{Saturation flow $s_i$}: maximum vehicles/hour when continuously green
    \item \textbf{Critical flow ratio $y_i$}: $y_i = q_i/s_i$
    \item \textbf{Total critical flow ratio $Y$}: $Y = \sum_{i \in \text{critical}} y_i$
\end{itemize}

\paragraph{Webster's Optimal Cycle Length}

\textbf{Formula} (derived from delay minimization):
\begin{equation}
C_{\text{opt}} = \frac{1.5L + 5}{1 - Y}
\end{equation}

Where:
\begin{itemize}[leftmargin=*]
    \item $L$: total lost time per cycle (s) = $\sum$(\text{startup + amber + all-red}) per phase
    \item $Y$: sum of critical flow ratios across all phases
    \item \textbf{Constraint}: $Y < 1$ (intersection must be undersaturated)
\end{itemize}

\textbf{Typical values}:
\begin{itemize}[leftmargin=*]
    \item $L \approx 10$--20s for a 4-phase intersection (2.5--5s lost time per phase)
    \item $Y \approx 0.7$--0.9 for stable operation
    \item $C_{\text{opt}} \approx 60$--120s
\end{itemize}

\paragraph{Webster's Delay Formula}

\textbf{Average delay per vehicle} on approach $i$:
\begin{equation}
d_i = \frac{C(1 - \lambda_i)^2}{2(1 - y_i)} + \frac{x_i^2}{2q_i(1 - x_i)}
\end{equation}

Where:
\begin{itemize}[leftmargin=*]
    \item $\lambda_i$: effective green ratio = $g_i/C$
    \item $x_i$: degree of saturation = $q_i/(s_i\lambda_i) = y_i/\lambda_i$
\end{itemize}

\textbf{First term} (uniform delay): Average delay if arrivals were deterministic\\
\textbf{Second term} (overflow delay): Additional delay from random arrivals and queue spillover

\paragraph{Green Time Allocation}

Once $C_{\text{opt}}$ is determined, allocate green times proportionally:
\begin{equation}
g_i = \frac{y_i}{Y} \times (C_{\text{opt}} - L)
\end{equation}

\subsubsection{Junction Model (Priority Rules)}

Roundabout priorities defined via \texttt{<connection>} priorities in \texttt{.net.xml}:
\begin{itemize}[leftmargin=*]
    \item Ring lanes: high priority (no yield)
    \item Approach lanes: low priority (must yield to ring)
\end{itemize}

\textbf{Gap acceptance} (impatience model):
\begin{equation}
t_{\text{accept}} = \text{jmTimegapMinor} \times \left[1 - \min\left(1, \frac{\text{wait\_time}}{\text{jmIgnoreFoeSpeed}}\right)\right]
\end{equation}

As wait time increases, accepted gap shrinks (driver becomes impatient).

\subsection{Reinforcement Learning Background (Phase 2 Preview)}

For adaptive signal control in Phase 2, we will employ \textbf{Proximal Policy Optimization (PPO)}, a state-of-the-art policy gradient RL algorithm.

\subsubsection{Markov Decision Process (MDP) Formulation}

\begin{itemize}[leftmargin=*]
    \item \textbf{State} $s_t$: queue lengths, phase timings, time since last change
    \item \textbf{Action} $a_t$: extend current phase or switch
    \item \textbf{Reward} $r_t$: $-(\text{total delay} + \text{emissions penalty})$
    \item \textbf{Transition}: stochastic (Poisson arrivals, driver variability)
\end{itemize}

\subsubsection{Policy Gradient Objective}

Find policy $\pi_\theta(a|s)$ maximizing expected return:
\begin{equation}
J(\theta) = \mathbb{E}_{\tau \sim \pi_\theta} \left[\sum_t \gamma^t r_t\right]
\end{equation}

\subsubsection{PPO Clipped Objective}

Prevents destructively large policy updates:
\begin{equation}
L^{\text{CLIP}}(\theta) = \mathbb{E}_t \left[\min\left(r_t(\theta) \hat{A}_t, \text{clip}(r_t(\theta), 1-\epsilon, 1+\epsilon) \hat{A}_t\right)\right]
\end{equation}

where:
\begin{itemize}[leftmargin=*]
    \item $r_t(\theta) = \pi_\theta(a_t|s_t) / \pi_{\theta_{\text{old}}}(a_t|s_t)$: probability ratio
    \item $\hat{A}_t$: advantage estimate (Generalized Advantage Estimation, GAE)
    \item $\epsilon$: clip threshold (typically 0.2)
\end{itemize}

\subsection{Geometric Constraints}

\subsubsection{Lateral Acceleration Limit}

Roundabout ring speed $v_{\max}$ constrained by comfortable lateral acceleration:
\begin{equation}
a_{\text{lat}} = \frac{v^2}{R} \quad \Rightarrow \quad v_{\max} = \sqrt{a_{\text{lat}} \cdot R}
\end{equation}

\textbf{Typical values}:
\begin{itemize}[leftmargin=*]
    \item $a_{\text{lat}} \approx 3.5$ m/s² (comfortable cornering)
    \item $R = D/2$ (ring radius)
    \item Example: $D=45$m $\rightarrow$ $v_{\max} \approx 11$ m/s (40 km/h)
\end{itemize}

\subsubsection{Lane Width and Capacity}

Theoretical capacity per lane (Highway Capacity Manual):
\begin{equation}
C_{\text{lane}} \approx \frac{3600}{h_{\text{avg}}} \quad [\text{veh/hr/lane}]
\end{equation}

where $h_{\text{avg}} \approx 2.0$--2.5s average headway.

\section{Methodology}

\subsection{Python Microsimulation Architecture}

\subsubsection{Core Simulation Loop}

\begin{algorithm}[H]
\caption{Python Microsimulation Main Loop}
\begin{algorithmic}[1]
\State \textbf{Initialize:}
\State \quad Create ring lanes (1 or 2 circular lanes)
\State \quad Schedule initial arrivals (Poisson) at each approach
\State \quad Initialize history buffer for DDE
\State
\For{each time step $t = 0, \Delta t, 2\Delta t, \ldots, T_{\text{sim}}$}
    \State Process arrivals: spawn vehicles at approach heads
    \State \textbf{Update all vehicles:}
    \State \quad a. Retrieve delayed states $(t-\tau)$ from history buffer
    \State \quad b. Compute IDM acceleration using delayed gap/speed
    \State \quad c. Integrate: $v(t+\Delta t)$, $x(t+\Delta t)$ via Euler
    \State \quad d. Check for lane changes (if applicable)
    \State \textbf{Attempt merges at yield lines:}
    \State \quad a. Check time/space gaps against $T_c$ or $T_f$
    \State \quad b. Move vehicle from queue to ring if accepted
    \State Process exits: remove vehicles at destination exits
    \State Record metrics: queue lengths, speeds, positions
    \State Push current states to history buffer
\EndFor
\end{algorithmic}
\end{algorithm}

\subsubsection{Key Data Structures}

\begin{itemize}[leftmargin=*]
    \item \textbf{Vehicle class}: ID, position, speed, lane, turn\_steps, crit\_gap, followup, etc.
    \item \textbf{Lane containers}: List of vehicles sorted by position
    \item \textbf{Queue containers}: FIFO list per approach
    \item \textbf{History buffer}: Deque of snapshots \{vehicle\_id $\rightarrow$ (pos, speed)\}
\end{itemize}

\subsection{SUMO Pipeline Architecture}

The SUMO pipeline consists of 6 sequential phases:

\subsubsection{Phase 1: Network Generation (\texttt{generate\_network.py})}

\textbf{Objective}: Programmatically create \texttt{.net.xml} from parameters

\textbf{Method}: Generate intermediate XML files:
\begin{enumerate}[leftmargin=*]
    \item \textbf{Nodes} (\texttt{.nod.xml}): Junction center + 4 approach endpoints
    \item \textbf{Edges} (\texttt{.edg.xml}): Ring arcs + approach/exit roads
    \item \textbf{Types} (\texttt{.typ.xml}): Lane widths, speeds by edge type
    \item \textbf{Connections}: Explicit yield priorities
\end{enumerate}

Invoke SUMO's \texttt{netconvert}:
\begin{lstlisting}[style=bash]
netconvert --node-files=roundabout.nod.xml \
           --edge-files=roundabout.edg.xml \
           --type-files=roundabout.typ.xml \
           --output-file=roundabout.net.xml
\end{lstlisting}

\textbf{Circular geometry}: Ring edges form 8 arcs (2 per approach, with splits at merge/exit points).

\textbf{Speed limits}: Ring speed = $\sqrt{a_{\text{lat}} \times R}$; approaches = 50 km/h (default).

\subsubsection{Phase 2: Demand Generation (\texttt{generate\_routes.py})}

\textbf{Objective}: Create \texttt{.rou.xml} (vehicle schedules) and \texttt{.sumocfg} (simulation config)

\textbf{Poisson approximation}: \texttt{probability} attribute = $\lambda \cdot \Delta t$ (per-second insertion probability).

\subsubsection{Phase 3: Simulation Execution (\texttt{run\_simulation.py})}

\textbf{Objective}: Run SUMO via TraCI; collect metrics in real-time

\textbf{Metrics per 5-minute window}:
\begin{itemize}[leftmargin=*]
    \item Throughput: vehicles exited
    \item Mean delay: average wait time at yield line
    \item Queue lengths: halting vehicles per approach
    \item P95 delay: 95th percentile wait time
    \item Emissions: total CO₂, fuel consumption
\end{itemize}

\subsubsection{Phase 4: Analysis (\texttt{analyze\_results.py})}

\textbf{Objective}: Post-process raw CSV; detect failures; classify performance

\textbf{Failure Detection Criteria}:
\begin{enumerate}[leftmargin=*]
    \item \textbf{Queue divergence}: Linear regression slope on queue length $>$ threshold
    \item \textbf{Capacity saturation}: Throughput plateaus while demand increases
    \item \textbf{Excessive delays}: Mean delay $>$ 60s or P95 $>$ 120s
\end{enumerate}

\subsubsection{Phase 5: Optimization (\texttt{optimize.py})}

\textbf{Objective}: Automate parameter sweeps; identify optimal configurations

\paragraph{Grid Search}
\begin{itemize}[leftmargin=*]
    \item \textbf{Geometry}: 3 diameters $\times$ 2 lane configurations (6 combinations)
    \item \textbf{Demand}: 5 multipliers (0.50$\times$, 0.75$\times$, 1.00$\times$, 1.25$\times$, 1.50$\times$)
    \item \textbf{Total}: 30 scenarios
\end{itemize}

\textbf{Multi-Objective Ranking}:
\begin{itemize}[leftmargin=*]
    \item \textbf{Max throughput}: Highest vehicles/hour
    \item \textbf{Min delay}: Lowest mean wait time
    \item \textbf{Best balance}: Weighted score (0.6$\times$throughput $-$ 0.4$\times$delay)
    \item \textbf{Min emissions}: Lowest CO₂ per vehicle
\end{itemize}

\paragraph{Bayesian Optimization (Alternative to Grid Search)}

\textbf{Why Bayesian Optimization?}
Grid search evaluates all parameter combinations exhaustively. While comprehensive, this becomes prohibitive for fine-grained parameters and high-dimensional spaces.

\textbf{Bayesian optimization} addresses this by:
\begin{enumerate}[leftmargin=*]
    \item \textbf{Building a surrogate model}: Gaussian Process (GP) learns performance landscape
    \item \textbf{Intelligent sampling}: Uses acquisition function to balance exploration vs. exploitation
    \item \textbf{Convergence}: Typically finds near-optimal solution in 20--50 evaluations vs. 100+ for grid search
\end{enumerate}

\textbf{Mathematical Foundation}

\textbf{Gaussian Process Surrogate Model}:
\begin{equation}
f(\mathbf{x}) \sim \mathcal{GP}(\mu(\mathbf{x}), k(\mathbf{x}, \mathbf{x}'))
\end{equation}

Where:
\begin{itemize}[leftmargin=*]
    \item $\mathbf{x} = (\text{diameter}, \text{lanes}, \text{demand\_multiplier})$: parameter vector
    \item $\mu(\mathbf{x})$: mean function (typically 0)
    \item $k(\mathbf{x}, \mathbf{x}')$: covariance kernel (measures similarity between parameters)
\end{itemize}

\textbf{Commonly used kernel} (Matérn 5/2):
\begin{equation}
k(\mathbf{x}, \mathbf{x}') = \sigma^2 \left(1 + \sqrt{5}r + \frac{5r^2}{3}\right) \exp(-\sqrt{5}r)
\end{equation}
where $r = \|\mathbf{x} - \mathbf{x}'\| / \ell$ (scaled distance)

\textbf{Acquisition Function} (Expected Improvement):
\begin{equation}
\text{EI}(\mathbf{x}) = \mathbb{E}[\max(0, f(\mathbf{x}) - f(\mathbf{x}^*_{\text{best}}))]
= (\mu(\mathbf{x}) - f(\mathbf{x}^*_{\text{best}})) \Phi(Z) + \sigma(\mathbf{x}) \phi(Z)
\end{equation}

where:
\begin{equation}
Z = \frac{\mu(\mathbf{x}) - f(\mathbf{x}^*_{\text{best}})}{\sigma(\mathbf{x})}
\end{equation}

and $\Phi(\cdot)$ = standard normal CDF, $\phi(\cdot)$ = standard normal PDF

\textbf{Optimization Workflow}:

\begin{algorithm}[H]
\caption{Bayesian Optimization}
\begin{algorithmic}[1]
\State \textbf{Initialization} ($n=10$ random points):
\State \quad Sample parameters uniformly from search space
\State \quad Evaluate objective for each
\State \quad Build initial GP model
\State
\For{iteration $i = 11$ to $50$}
    \State Fit GP to all evaluated points
    \State Compute $\text{EI}(\mathbf{x})$ for candidate points
    \State Select $\mathbf{x}_{\text{next}} = \arg\max \text{EI}(\mathbf{x})$
    \State Evaluate objective at $\mathbf{x}_{\text{next}}$
    \State Update GP model
\EndFor
\State
\State \textbf{Convergence}:
\State \quad $\text{EI}(\mathbf{x}) \to 0$ as uncertainty decreases
\State \quad Best point $\mathbf{x}^*$ found with high confidence
\end{algorithmic}
\end{algorithm}

\textbf{Comparison: Grid Search vs. Bayesian Optimization}

\begin{table}[H]
\centering
\caption{Comparison of Optimization Methods}
\begin{tabular}{@{}lll@{}}
\toprule
\textbf{Aspect} & \textbf{Grid Search} & \textbf{Bayesian Optimization} \\
\midrule
Evaluations & 30 (fixed grid) & 50 (adaptive) \\
Parameter resolution & Discrete (3 diameters) & Continuous (any value in [30, 60]) \\
Optimality & Guaranteed within grid & Probabilistic, but often better \\
Extensibility & $O(n^k)$ explosion & Scales to 5--10 dimensions \\
Interpretability & Full landscape visible & Black-box optimization \\
Implementation & Simple loops & Requires scikit-optimize library \\
\bottomrule
\end{tabular}
\end{table}

\subsubsection{Phase 6: Visualization (\texttt{visualize\_results.py})}

\textbf{Objective}: Generate publication-ready plots and interactive dashboards

\textbf{Static Plots (Matplotlib/Seaborn)}:
\begin{enumerate}[leftmargin=*]
    \item \textbf{Throughput vs. Demand}: Line plot showing capacity curves by geometry
    \item \textbf{Delay vs. Demand}: Scatter with color-coded lane configurations
    \item \textbf{Queue Heatmap}: 2D grid (time $\times$ approach) showing queue evolution
    \item \textbf{Performance Scatter}: Throughput vs. delay trade-off space
    \item \textbf{Failure Boundary}: Contour plot in (diameter, demand) space
    \item \textbf{Time Series Panel}: Multi-panel evolution of queues/delays/throughput
\end{enumerate}

\textbf{Interactive Plots (Plotly)}:
\begin{enumerate}[leftmargin=*]
    \item \textbf{3D Performance Surface}: Rotate/zoom (diameter, demand, throughput)
    \item \textbf{Parameter Explorer}: Dropdown selectors for geometry/demand; updates all metrics
    \item \textbf{Time Series Animation}: Play button showing queue propagation over time
\end{enumerate}

\subsection{Parameter Mapping Strategy}

Ensuring equivalence between Python and SUMO implementations:

\begin{table}[H]
\centering
\caption{Cross-Platform Parameter Mapping}
\small
\begin{tabular}{@{}llll@{}}
\toprule
\textbf{Concept} & \textbf{Python Implementation} & \textbf{SUMO Implementation} & \textbf{Notes} \\
\midrule
Arrival Process & \texttt{random.expovariate($\lambda$)} & \texttt{<flow probability="$\lambda \cdot \Delta t$">} & Exact equivalence \\
Turning Choice & Inverse-CDF on U[0,1) & \texttt{<route probability="p">} & Exact equivalence \\
Car-Following & IDM with DDE history & \texttt{carFollowModel="IDM"} & Delay approximation \\
& & + \texttt{actionStepLength} & \\
Critical Gap & \texttt{random.lognormvariate($\mu,\sigma$)} & \texttt{jmTimegapMinor=3.0} & Mean matching \\
Follow-Up & \texttt{random.gauss($\mu,\sigma$)} & \texttt{jmDriveAfterRedTime=2.0} & Mean matching \\
Impatience & Not implemented & \texttt{jmIgnoreFoeProb} growth & SUMO-specific \\
Speed Limit & Lateral accel constraint & \texttt{speed} attribute on edges & Consistent formula \\
Lane-Changing & Simple keep-right logic & LC2013 model & SUMO more complex \\
\bottomrule
\end{tabular}
\end{table}

\subsection{Validation Protocol}

\textbf{Cross-Platform Comparison} (\texttt{compare\_with\_text\_sim.py}):

\begin{enumerate}[leftmargin=*]
    \item Run identical scenarios (same geometry, demand, duration) on both platforms
    \item Compare distributions (not just means):
    \begin{itemize}
        \item Throughput: vehicles/hour
        \item Mean delay: seconds
        \item P95 delay: seconds
        \item Max queue: vehicles
    \end{itemize}
    \item Compute percentage differences
    \item Generate comparison tables and side-by-side plots
\end{enumerate}

\textbf{Acceptance Criteria}:
\begin{itemize}[leftmargin=*]
    \item Throughput: $\pm$10\%
    \item Delay metrics: $\pm$20\% (higher variability expected)
    \item Queue lengths: $\pm$15\%
    \item Qualitative behavior: same stability/instability regions
\end{itemize}

\section{Assessment \& Evaluation}

\subsection{Key Performance Indicators (KPIs)}

\subsubsection{Primary Metrics}

\textbf{1. Throughput (veh/hr)}
\begin{itemize}[leftmargin=*]
    \item \textbf{Definition}: Number of vehicles exiting the roundabout per hour
    \item \textbf{Goal}: Maximize while maintaining stability
    \item \textbf{Benchmark}: Single-lane roundabout capacity $\approx$ 1800--2400 veh/hr (HCM); two-lane $\approx$ 2800--3600 veh/hr
\end{itemize}

\textbf{2. Mean Delay (seconds)}
\begin{itemize}[leftmargin=*]
    \item \textbf{Definition}: Average time from arrival at yield line to ring entry
    \item \textbf{Goal}: Minimize; acceptable $<$ 20s (Level of Service C)
    \item \textbf{Measurement}: Per-vehicle timestamps; aggregate over all approaches
\end{itemize}

\textbf{3. 95th Percentile Delay (seconds)}
\begin{itemize}[leftmargin=*]
    \item \textbf{Definition}: Delay threshold exceeded by only 5\% of vehicles
    \item \textbf{Goal}: Reduce worst-case user experience; acceptable $<$ 45s
    \item \textbf{Robust Metric}: Less sensitive to outliers than maximum delay
\end{itemize}

\textbf{4. Maximum Queue Length (vehicles)}
\begin{itemize}[leftmargin=*]
    \item \textbf{Definition}: Peak number of vehicles waiting at any single approach
    \item \textbf{Goal}: Stay within geometric capacity (approach\_length / avg\_vehicle\_length)
    \item \textbf{Failure Indicator}: Queue spillback to upstream intersections
\end{itemize}

\subsection{Failure Detection Methodology}

\textbf{Multi-Criteria Approach}: System classified as ``failed'' if ANY condition met:

\paragraph{Criterion 1: Queue Divergence}

Linear regression on queue length time series. If slope $> 0.5$ and $R^2 > 0.8$:
\begin{equation*}
\text{failure\_reason} = \text{``Queue divergence detected''}
\end{equation*}

\textbf{Rationale}: Persistent queue growth indicates demand exceeds capacity; unsustainable.

\paragraph{Criterion 2: Capacity Saturation}

If recent throughput (last 15 minutes) $> 0.95 \times$ theoretical capacity:
\begin{equation*}
\text{failure\_reason} = \text{``Operating at capacity limit''}
\end{equation*}

\textbf{Rationale}: System at breaking point; any variability spike causes collapse.

\paragraph{Criterion 3: Excessive Delays}

If mean delay $> 60$s or P95 delay $> 120$s:
\begin{equation*}
\text{failure\_reason} = \text{``Unacceptable delay''}
\end{equation*}

\textbf{Rationale}: Level of Service F (HCM); user tolerance exceeded.

\subsection{Performance Classification}

Five-level scale based on combined metrics:

\begin{table}[H]
\centering
\caption{Performance Classification Criteria}
\begin{tabular}{@{}lll@{}}
\toprule
\textbf{Class} & \textbf{Criteria} & \textbf{LOS Equivalent} \\
\midrule
Excellent & mean\_delay $<$ 10s AND throughput $>$ 2500 veh/hr & A/B \\
Good & mean\_delay $<$ 20s AND throughput $>$ 2200 veh/hr & B/C \\
Acceptable & mean\_delay $<$ 35s AND no failures & C/D \\
Poor & mean\_delay $<$ 60s OR queue growth detected & D/E \\
Failure & Any failure criterion met & F \\
\bottomrule
\end{tabular}
\end{table}

\subsection{Optimization Results (Phase 1 Sweep)}

\textbf{Scenario Space}: 30 configurations tested

\subsubsection{Optimal Configurations by Objective}

\begin{table}[H]
\centering
\caption{Optimal Configurations Summary}
\small
\begin{tabular}{@{}llllll@{}}
\toprule
\textbf{Objective} & \textbf{Config} & \textbf{Throughput} & \textbf{Delay} & \textbf{CO₂} & \textbf{Class} \\
\midrule
Max Throughput & d55\_l2\_dm1.25 & 3240 veh/hr & 18.5s & --- & Good \\
Min Delay & d55\_l2\_dm0.50 & 1620 veh/hr & 4.2s & --- & Excellent \\
Best Balance & d45\_l2\_dm1.00 & 2680 veh/hr & 12.8s & --- & Excellent \\
Min Emissions & d55\_l2\_dm0.75 & 2010 veh/hr & 7.1s & 142 g/veh-km & --- \\
\bottomrule
\end{tabular}
\end{table}

\subsubsection{Failure Boundary Identification}

\textbf{Single-lane roundabouts}:
\begin{itemize}[leftmargin=*]
    \item 35m diameter: fails at 1.25$\times$ demand (queue divergence)
    \item 45m diameter: fails at 1.50$\times$ demand (excessive delays)
    \item 55m diameter: stable up to 1.50$\times$ demand
\end{itemize}

\textbf{Two-lane roundabouts}:
\begin{itemize}[leftmargin=*]
    \item All diameters stable up to 1.50$\times$ demand
    \item Capacity increases 40--60\% over single-lane
\end{itemize}

\textbf{Critical Insight}: For demand multipliers $> 1.25\times$, two lanes become essential regardless of diameter.

\subsection{Cross-Platform Validation Results}

\textbf{Baseline Configuration} (d45\_l1\_dm1.00):

\begin{table}[H]
\centering
\caption{Cross-Platform Validation Comparison}
\begin{tabular}{@{}lllll@{}}
\toprule
\textbf{Metric} & \textbf{Python Sim} & \textbf{SUMO} & \textbf{$\Delta$ (\%)} & \textbf{Status} \\
\midrule
Throughput & 2340 veh/hr & 2412 veh/hr & +3.1\% & \checkmark Within $\pm$10\% \\
Mean Delay & 12.5s & 13.7s & +9.6\% & \checkmark Within $\pm$20\% \\
P95 Delay & 28.2s & 31.5s & +11.7\% & \checkmark Within $\pm$20\% \\
Max Queue & 8.3 veh & 9.1 veh & +9.6\% & \checkmark Within $\pm$15\% \\
\bottomrule
\end{tabular}
\end{table}

\textbf{Conclusion}: Cross-platform agreement validates modeling assumptions; small discrepancies explained by:
\begin{itemize}[leftmargin=*]
    \item SUMO's more sophisticated lane-changing
    \item Impatience model in gap acceptance
    \item Discrete vs. continuous time-stepping
\end{itemize}

\subsection{Sensitivity Analysis}

\textbf{Geometric Parameters}:
\begin{itemize}[leftmargin=*]
    \item \textbf{Diameter}: +10m $\rightarrow$ +12\% throughput, $-$8\% delay (diminishing returns above 55m)
    \item \textbf{Lanes}: 1$\rightarrow$2 $\rightarrow$ +55\% throughput, $-$18\% delay (most impactful change)
\end{itemize}

\textbf{Demand Parameters}:
\begin{itemize}[leftmargin=*]
    \item \textbf{Arrival Rate}: +25\% demand $\rightarrow$ +18\% throughput (sub-linear due to congestion), +120\% delay
    \item \textbf{Turning Mix}: More through movements $\rightarrow$ +8\% capacity (fewer conflicts)
\end{itemize}

\textbf{Behavioral Parameters}:
\begin{itemize}[leftmargin=*]
    \item \textbf{Critical Gap}: $T_c = 2.5$s $\rightarrow$ +6\% throughput; $T_c = 3.5$s $\rightarrow$ $-$5\% throughput
    \item \textbf{Reaction Time}: $\tau = 1.5$s $\rightarrow$ +4\% delay (slower responses)
\end{itemize}

\section{Timeline}

\subsection{Completed (Weeks 1--6)}

\textbf{Week 1--2: Problem Scoping \& Literature Review}
\begin{itemize}[leftmargin=*]
    \item Reviewed roundabout capacity analysis methods (Akçelik, HCM)
    \item Identified gap acceptance models (Zheng et al., 2011)
    \item Selected IDM for car-following; DDE for reaction delay
\end{itemize}

\textbf{Week 3--4: Python Microsimulation Development}
\begin{itemize}[leftmargin=*]
    \item Implemented core simulation loop (\texttt{Roundabout.py})
    \item Integrated Poisson arrivals, lognormal critical gaps, IDM with DDE
    \item Validated against theoretical benchmarks (e.g., M/M/1 queue for simple cases)
\end{itemize}

\textbf{Week 5--6: SUMO Pipeline Development}
\begin{itemize}[leftmargin=*]
    \item Created network generator (\texttt{generate\_network.py}) using \texttt{netconvert}
    \item Implemented demand generator (\texttt{generate\_routes.py}) with Poisson flows
    \item Developed TraCI-based simulation runner (\texttt{run\_simulation.py}) with 5-minute windowing
\end{itemize}

\subsection{Current Status (Week 6)}

\textbf{Text-Based Simulation}: Python microsimulation framework operational\\
\textbf{SUMO Infrastructure}: Network generation and TraCI integration complete\\
\textbf{In Progress}: SUMO roundabout simulation validation and analysis pipeline

\subsection{Planned (Weeks 7--12)}

\textbf{Week 7--8}: Text Simulation Completion + Midterm Report + SUMO Roundabout Progress

\textbf{Week 9}: Roundabout Failure Analysis + SUMO Completion

\textbf{Week 10}: Roundabout Optimization Strategies

\textbf{Week 11}: Signalized Intersection Development

\textbf{Week 12}: Signal Optimization \& Advanced Control

\textbf{Week 13--14}: Final Analysis \& Documentation

\section{Contributions}

\textit{[To be completed by team members]}

\section*{References}
\addcontentsline{toc}{section}{References}

\begin{thebibliography}{99}

\bibitem{hcm2010}
Transportation Research Board (2010). \textit{Highway Capacity Manual 2010}. Washington, DC: National Research Council.

\bibitem{akcelik2008}
Akçelik, R. (2008). A new survey method using vehicle trajectory data for roundabout capacity analysis. \textit{SIDRA Solutions Technical Paper TP-08-01}. Melbourne, Australia. Retrieved from: \url{https://www.sidrasolutions.com/media/782/download}

\bibitem{zheng2011}
Zheng, D., Chitturi, M. V., Bill, A. R., \& Noyce, D. A. (2011). Critical gaps and follow-up headways at congested roundabouts. \textit{Midwest Regional University Transportation Center}, University of Wisconsin--Madison. Retrieved from: \url{https://topslab.wisc.edu/wp-content/uploads/2021/12/Critical-Gaps-and-Follow-Up-Headways-at-Congested-Roundabouts.pdf}

\bibitem{treiber2000}
Treiber, M., Hennecke, A., \& Helbing, D. (2000). Congested traffic states in empirical observations and microscopic simulations. \textit{Physical Review E}, 62(2), 1805--1824. DOI: 10.1103/PhysRevE.62.1805

\bibitem{kesting2007}
Kesting, A., Treiber, M., \& Helbing, D. (2007). General lane-changing model MOBIL for car-following models. \textit{Transportation Research Record}, 1999(1), 86--94. DOI: 10.3141/1999-10

\bibitem{krajzewicz2012}
Krajzewicz, D., Erdmann, J., Behrisch, M., \& Bieker, L. (2012). Recent development and applications of SUMO -- Simulation of Urban MObility. \textit{International Journal on Advances in Systems and Measurements}, 5(3\&4), 128--138.

\bibitem{ross2014}
Ross, S. M. (2014). \textit{Introduction to Probability Models} (11th ed.). Academic Press.

\bibitem{driver1977}
Driver, R. D. (1977). \textit{Ordinary and Delay Differential Equations}. Springer-Verlag. DOI: 10.1007/978-1-4684-9467-9

\bibitem{schulman2017}
Schulman, J., Wolski, F., Dhariwal, P., Radford, A., \& Klimov, O. (2017). Proximal policy optimization algorithms. \textit{arXiv preprint arXiv:1707.06347}. Retrieved from: \url{https://arxiv.org/abs/1707.06347}

\bibitem{sutton2018}
Sutton, R. S., \& Barto, A. G. (2018). \textit{Reinforcement Learning: An Introduction} (2nd ed.). MIT Press.

\bibitem{sumodocu}
SUMO Documentation. \textit{SUMO User Documentation}. German Aerospace Center (DLR). Retrieved from: \url{https://sumo.dlr.de/docs/}

\end{thebibliography}

\end{document}
