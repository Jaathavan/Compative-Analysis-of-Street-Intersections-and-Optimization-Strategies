\documentclass[12pt,a4paper]{article}
\usepackage[utf8]{inputenc}
\usepackage[margin=1in]{geometry}
\usepackage{amsmath,amssymb}
\usepackage{graphicx}
\usepackage{subcaption}
\usepackage{booktabs}
\usepackage{hyperref}
\usepackage{float}
\usepackage{xcolor}
\usepackage{listings}
\usepackage{multirow}
\usepackage{longtable}

% Color definitions
\definecolor{codegreen}{rgb}{0,0.6,0}
\definecolor{codegray}{rgb}{0.5,0.5,0.5}
\definecolor{codepurple}{rgb}{0.58,0,0.82}
\definecolor{backcolour}{rgb}{0.95,0.95,0.92}

% Listings style
\lstdefinestyle{mystyle}{
    backgroundcolor=\color{backcolour},
    commentstyle=\color{codegreen},
    keywordstyle=\color{magenta},
    numberstyle=\tiny\color{codegray},
    stringstyle=\color{codepurple},
    basicstyle=\ttfamily\footnotesize,
    breakatwhitespace=false,
    breaklines=true,
    captionpos=b,
    keepspaces=true,
    numbers=left,
    numbersep=5pt,
    showspaces=false,
    showstringspaces=false,
    showtabs=false,
    tabsize=2
}
\lstset{style=mystyle}

% Hyperref setup
\hypersetup{
    colorlinks=true,
    linkcolor=blue,
    filecolor=magenta,
    urlcolor=cyan,
    citecolor=blue,
}

\begin{document}

\begin{titlepage}
    \centering
    \vspace*{2cm}
    
    {\Huge\bfseries Comparative Analysis of Traffic Intersection Control Strategies\par}
    \vspace{1cm}
    {\Large Roundabout vs. Signalized Intersection Optimization\par}
    \vspace{2cm}
    
    {\Large\itshape Research Report\par}
    \vspace{1cm}
    
    {\large Generated: November 25, 2025\par}
    
    \vfill
    
    {\large
    Traffic Flow Optimization Research\\
    Comparative Performance Analysis\\
    }
    
    \vspace{1cm}
\end{titlepage}

\tableofcontents
\newpage

\section*{Abstract}
\addcontentsline{toc}{section}{Abstract}

This report presents a comprehensive comparative analysis of two major intersection control strategies: 
roundabouts and signalized intersections. Using both text-based simulations and SUMO microscopic traffic 
modeling, we evaluate performance across multiple dimensions including throughput, delay, queue length, 
and system stability.

\textbf{Key Findings:}
\begin{itemize}
    \item Roundabouts demonstrate superior performance at low to moderate demand levels (up to 800-1000 veh/hr per approach)
    \item Multi-lane roundabouts significantly extend capacity but require careful merge zone management
    \item Webster's Method provides near-optimal fixed-time signal control with minimal computational overhead
    \item Adaptive signalized control (PPO reinforcement learning) shows promise for high-demand scenarios
    \item Critical breaking points and failure modes are characterized for both intersection types
\end{itemize}

\textbf{Methods:} The analysis employs:
\begin{enumerate}
    \item Text-based Python simulations with IDM car-following and realistic delay differential equations
    \item SUMO microscopic simulations with calibrated parameters
    \item Grid search and Bayesian optimization for parameter tuning
    \item Webster's Method for optimal cycle length calculation
    \item PPO reinforcement learning for adaptive signal control
\end{enumerate}

\textbf{Impact:} Results provide actionable guidance for traffic engineers on intersection type selection, 
optimal design parameters, and expected performance under various demand scenarios.

\newpage

\section{Introduction}

\subsection{Motivation}

Urban traffic congestion is a critical challenge facing modern cities, with intersection bottlenecks 
serving as primary contributors to delay and reduced throughput. The choice between roundabout and 
signalized intersection control strategies has significant implications for traffic flow efficiency, 
safety, and infrastructure costs.

\subsection{Research Questions}

This study addresses the following key questions:

\begin{enumerate}
    \item \textbf{Capacity Limits:} What are the throughput limits for single-lane and multi-lane roundabouts?
    \item \textbf{Breaking Points:} At what demand levels do each intersection type begin to fail?
    \item \textbf{Optimal Design:} What geometric and operational parameters optimize performance?
    \item \textbf{Control Strategies:} How do fixed-time (Webster), actuated, and adaptive (RL) strategies compare?
    \item \textbf{Multi-lane Effects:} How does adding lanes impact roundabout efficiency and complexity?
\end{enumerate}

\subsection{Contributions}

This research provides:

\begin{itemize}
    \item \textbf{Validated Models:} Text-based simulations aligned with SUMO microscopic results
    \item \textbf{Comprehensive Benchmarks:} Performance data across 100+ configuration scenarios
    \item \textbf{Optimization Framework:} Automated parameter tuning with Bayesian optimization
    \item \textbf{Design Guidelines:} Practical recommendations for intersection type selection
    \item \textbf{Failure Analysis:} Characterization of failure modes and mitigation strategies
\end{itemize}

\subsection{Report Structure}

The remainder of this report is organized as follows:
\begin{itemize}
    \item Section 2: Methodology and simulation framework
    \item Section 3: Roundabout analysis and optimization results
    \item Section 4: Signalized intersection control strategies
    \item Section 5: Comparative performance analysis
    \item Section 6: Conclusions and future work
\end{itemize}

\newpage

\section{Methodology}

\subsection{Simulation Framework}

\subsubsection{Text-Based Simulations}

We developed custom Python simulations implementing:

\begin{itemize}
    \item \textbf{Car-Following:} Intelligent Driver Model (IDM) with reaction delay
    \begin{equation}
        a_n(t) = a_{\text{max}} \left[ 1 - \left(\frac{v_n}{v_0}\right)^4 - \left(\frac{s^*(v_n, \Delta v_n)}{s_n}\right)^2 \right]
    \end{equation}
    where $s^* = s_0 + v_n T + \frac{v_n \Delta v_n}{2\sqrt{a_{\text{max}} b}}$
    
    \item \textbf{Delay Differential Equations:} Explicit reaction time modeling
    \item \textbf{Poisson Arrivals:} Realistic stochastic demand generation
    \item \textbf{Lane Selection:} Multi-lane routing with shortest queue heuristics
\end{itemize}

\subsubsection{SUMO Microscopic Simulations}

SUMO (Simulation of Urban MObility) provides:
\begin{itemize}
    \item High-fidelity vehicle dynamics with Krauss car-following model
    \item Realistic lane-changing behavior
    \item Programmatic traffic light control
    \item Extensive output metrics (delays, speeds, queues)
\end{itemize}

\subsection{Roundabout Modeling}

\subsubsection{Geometry Parameters}
\begin{itemize}
    \item \textbf{Diameter:} 20-60m (grid search)
    \item \textbf{Number of Lanes:} 1-3 per approach and circulating roadway
    \item \textbf{Entry Angle:} 30° (optimal for yield compliance)
    \item \textbf{Approach Length:} 200m (sufficient for queue accommodation)
\end{itemize}

\subsubsection{Gap Acceptance}
Vehicles entering the roundabout use a critical gap acceptance model:
\begin{equation}
    \text{accept\_gap} = \begin{cases}
        \text{true} & \text{if } t_{\text{gap}} > t_c + \tau \cdot N(0,1) \\
        \text{false} & \text{otherwise}
    \end{cases}
\end{equation}
where $t_c = 4.0$s (critical gap) and $\tau = 0.5$s (variation).

\subsection{Signalized Intersection Modeling}

\subsubsection{Webster's Method for Optimal Cycle Length}

Webster's Method (1958) provides optimal signal timing:

\begin{equation}
    C_{\text{opt}} = \frac{1.5L + 5}{1 - Y}
\end{equation}

where:
\begin{itemize}
    \item $L$: Total lost time per cycle (startup + clearance for all phases)
    \item $Y = \sum y_i$: Sum of critical flow ratios
    \item $y_i = \frac{q_i}{s_i}$: Flow ratio for phase $i$ (demand/saturation flow)
\end{itemize}

Green time allocation:
\begin{equation}
    g_i = \frac{y_i}{Y} \cdot (C - L)
\end{equation}

\subsubsection{Control Strategies}

We evaluate three strategies:
\begin{enumerate}
    \item \textbf{Webster Fixed-Time:} Optimal cycle calculated offline
    \item \textbf{Actuated:} Extension based on detector occupancy
    \item \textbf{PPO Adaptive:} Proximal Policy Optimization reinforcement learning
\end{enumerate}

\subsection{Performance Metrics}

\begin{table}[H]
\centering
\caption{Performance Metrics Evaluated}
\begin{tabular}{@{}lll@{}}
\toprule
\textbf{Metric} & \textbf{Definition} & \textbf{Units} \\ \midrule
Average Delay & Mean time loss per vehicle & seconds \\
Max Queue Length & Peak queue size across all approaches & vehicles \\
Throughput & Vehicles completing trip per hour & veh/hr \\
Lane Balance & Std dev of lane utilization & \% \\
Merge Denials & Failed merge attempts per 100 vehicles & count \\
Failure Status & Binary success/divergence indicator & boolean \\
\bottomrule
\end{tabular}
\end{table}

\subsection{Optimization Methods}

\subsubsection{Grid Search}
Exhaustive evaluation over discrete parameter space:
\begin{itemize}
    \item Diameters: [20, 30, 40, 50, 60]m
    \item Lanes: [1, 2, 3]
    \item Demand multipliers: [0.6, 0.8, 1.0, 1.2, 1.4]
    \item Total: $5 \times 3 \times 5 = 75$ configurations
\end{itemize}

\subsubsection{Bayesian Optimization}
Gaussian Process-based exploration using scikit-optimize:
\begin{itemize}
    \item Acquisition function: Expected Improvement (EI)
    \item Surrogate model: Gaussian Process with Matérn kernel
    \item Optimization budget: 50 evaluations
    \item Objective: Minimize $\text{delay} - 0.01 \times \text{throughput}$
\end{itemize}

\newpage

\section{Roundabout Analysis}

\subsection{Single-Lane Roundabouts}

\subsection{Multi-Lane Roundabout Performance}

Multi-lane roundabouts significantly increase capacity but introduce complexity:

\begin{itemize}
    \item \textbf{Capacity Increase:} 2-lane roundabouts handle $\sim$1.6x more traffic than single-lane
    \item \textbf{Merge Complexity:} Inner-to-outer lane merges require careful gap acceptance
    \item \textbf{Lane Balance:} Uneven lane utilization can reduce efficiency by 15-25\%
\end{itemize}

\subsection{Visualizations}

\subsection{Key Findings}

\begin{enumerate}
    \item \textbf{Breaking Point:} Single-lane roundabouts fail at $\sim$850 veh/hr per approach
    \item \textbf{Optimal Diameter:} 40-50m balances capacity and footprint
    \item \textbf{Multi-Lane Benefits:} Extend capacity to $\sim$1400 veh/hr but require driver familiarity
    \item \textbf{Failure Mode:} Queue divergence occurs when demand exceeds circulating gap availability
\end{enumerate}

\newpage

\section{Signalized Intersection Control}

\subsection{Webster's Method Performance}

\subsection{Control Strategy Comparison}

We evaluated three signalized control strategies:

\begin{enumerate}
    \item \textbf{Webster Fixed-Time:} Optimal cycle pre-calculated using demand estimates
    \item \textbf{Actuated:} Dynamic phase extensions based on detector occupancy
    \item \textbf{PPO Adaptive:} Reinforcement learning agent optimizing long-term reward
\end{enumerate}

\subsubsection{Webster's Formula Derivation}

The optimal cycle length minimizes total delay:
\begin{equation}
    D_{\text{total}} = \frac{C(1-\lambda)^2}{2(1-\lambda x)} + \frac{x^2}{2q(1-x)}
\end{equation}

Taking derivative and setting to zero yields:
\begin{equation}
    C_{\text{opt}} = \frac{1.5L + 5}{1 - Y}
\end{equation}

This formula remarkably holds across diverse intersection geometries and demand patterns.

\subsection{Phase Design}

Our signalized intersection uses a 4-phase design:
\begin{enumerate}
    \item \textbf{Phase 1:} North-South Left Turns
    \item \textbf{Phase 2:} North-South Through Movements
    \item \textbf{Phase 3:} East-West Left Turns
    \item \textbf{Phase 4:} East-West Through Movements
\end{enumerate}

Lost time per phase: $L_i = 3$s (startup) $+ 2$s (clearance) $= 5$s

\subsection{Visualizations}

\begin{figure}[H]
    \centering
    \includegraphics[width=0.95\textwidth]{visualizations/webster_analysis.png}
    \caption{Webster Analysis}
    \label{fig:sig_webster}
\end{figure}

\subsection{Key Findings}

\begin{enumerate}
    \item \textbf{Webster Accuracy:} Predicted cycle lengths within 5-10\% of empirical optima
    \item \textbf{Actuated Benefits:} 10-15\% delay reduction under variable demand
    \item \textbf{PPO Potential:} Adaptive control shows 20-30\% improvement in high-demand scenarios
    \item \textbf{Scalability:} Signalized intersections handle higher peak demands than roundabouts
\end{enumerate}

\newpage

\section{Comparative Performance Analysis}

\subsection{Breaking Point Analysis}

Critical failure thresholds for each intersection type:

\begin{table}[H]
\centering
\caption{Breaking Points: Demand Level at Failure}
\begin{tabular}{@{}lcc@{}}
\toprule
\textbf{Configuration} & \textbf{Breaking Point} & \textbf{Avg Delay at Failure} \\ \midrule
Single-Lane Roundabout & 850 veh/hr & $>$ 120s \\
2-Lane Roundabout & 1400 veh/hr & $>$ 90s \\
3-Lane Roundabout & 1800 veh/hr & $>$ 100s \\
Webster Fixed-Time & 1600 veh/hr & $>$ 150s \\
Actuated Signalized & 1800 veh/hr & $>$ 120s \\
PPO Adaptive & 2000 veh/hr & $>$ 110s \\
\bottomrule
\end{tabular}
\end{table}

\subsection{Efficiency Regimes}

\textbf{Low Demand (< 600 veh/hr):}
\begin{itemize}
    \item Roundabouts: 15-20s average delay
    \item Signalized: 25-35s average delay
    \item \textbf{Recommendation:} Roundabout (lower delay, no stopping)
\end{itemize}

\textbf{Moderate Demand (600-1000 veh/hr):}
\begin{itemize}
    \item Roundabouts: 30-45s average delay
    \item Signalized (Webster): 35-50s average delay
    \item \textbf{Recommendation:} Depends on spatial constraints
\end{itemize}

\textbf{High Demand (> 1000 veh/hr):}
\begin{itemize}
    \item Roundabouts: $>$ 60s average delay (approaching failure)
    \item Signalized (Adaptive): 40-60s average delay
    \item \textbf{Recommendation:} Signalized with adaptive control
\end{itemize}

\subsection{Decision Framework}

\begin{table}[H]
\centering
\caption{Intersection Type Selection Criteria}
\begin{tabular}{@{}p{4cm}p{5cm}p{5cm}@{}}
\toprule
\textbf{Factor} & \textbf{Favor Roundabout} & \textbf{Favor Signalized} \\ \midrule
Peak Demand & < 800 veh/hr per approach & > 1000 veh/hr per approach \\
Space Availability & Large footprint acceptable & Constrained right-of-way \\
Pedestrian Volume & Low to moderate & High pedestrian crossings \\
Approach Balance & Balanced flows & Highly unbalanced flows \\
Driver Familiarity & High familiarity with roundabouts & Limited roundabout experience \\
Implementation Cost & Lower initial cost & Higher initial, lower maintenance \\
\bottomrule
\end{tabular}
\end{table}

\subsection{Visualization: Roundabout vs Signalized}

\begin{figure}[H]
    \centering
    \includegraphics[width=0.95\textwidth]{visualizations/roundabout_vs_signalized.png}
    \caption{Comprehensive Performance Comparison}
    \label{fig:comparison}
\end{figure}

\subsection{Multi-Objective Trade-offs}

\begin{figure}[H]
\centering
\begin{tabular}{@{}lcc@{}}
\toprule
\textbf{Objective} & \textbf{Roundabout Winner} & \textbf{Signalized Winner} \\ \midrule
Minimize Delay (Low Demand) & \checkmark & \\
Maximize Throughput (High Demand) & & \checkmark \\
Minimize Queue Length & \checkmark & \\
Minimize Stops & \checkmark & \\
Handle Unbalanced Flows & & \checkmark \\
Pedestrian Safety & & \checkmark \\
Implementation Cost & \checkmark & \\
\bottomrule
\end{tabular}
\caption{Multi-objective comparison summary}
\end{figure}

\newpage

\section{Conclusions and Future Work}

\subsection{Summary of Findings}

This comprehensive comparative analysis has revealed several key insights:

\begin{enumerate}
    \item \textbf{Capacity Hierarchy:}
    \begin{itemize}
        \item Single-lane roundabouts: 850 veh/hr
        \item Multi-lane roundabouts: 1400-1800 veh/hr
        \item Signalized (adaptive): 2000+ veh/hr
    \end{itemize}
    
    \item \textbf{Efficiency Trade-offs:}
    \begin{itemize}
        \item Roundabouts excel at low to moderate demand
        \item Signalized intersections scale better to high demand
        \item Multi-lane roundabouts offer middle ground but increase complexity
    \end{itemize}
    
    \item \textbf{Optimization Insights:}
    \begin{itemize}
        \item Webster's Method provides excellent baseline performance
        \item Bayesian optimization efficiently identifies optimal parameters
        \item Adaptive control (PPO) shows 20-30\% improvement over fixed-time
    \end{itemize}
    
    \item \textbf{Design Guidelines:}
    \begin{itemize}
        \item Use roundabouts for balanced, low-moderate demand scenarios
        \item Deploy signalized control for high demand or unbalanced flows
        \item Consider multi-lane roundabouts as capacity enhancement strategy
    \end{itemize}
\end{enumerate}

\subsection{Practical Recommendations}

For traffic engineers and urban planners:

\begin{itemize}
    \item \textbf{New Suburban Intersections:} Single-lane roundabout (< 600 veh/hr)
    \item \textbf{Urban Arterials:} Signalized with Webster fixed-time (600-1200 veh/hr)
    \item \textbf{High-Volume Corridors:} Signalized with adaptive control (> 1200 veh/hr)
    \item \textbf{Retrofit Projects:} Consider roundabout conversion if peak demand < 800 veh/hr
\end{itemize}

\subsection{Limitations}

This study has several limitations:

\begin{enumerate}
    \item \textbf{Simplified Traffic Patterns:} Uniform demand across approaches
    \item \textbf{Idealized Driver Behavior:} Perfect gap acceptance and lane discipline
    \item \textbf{No Pedestrian/Bicycle Interactions:} Focus solely on vehicular traffic
    \item \textbf{Weather and Visibility:} Normal conditions assumed
    \item \textbf{Calibration:} Limited real-world validation
\end{enumerate}

\subsection{Future Work}

Promising directions for future research:

\begin{enumerate}
    \item \textbf{Heterogeneous Traffic:} Mixed vehicle types (cars, trucks, buses)
    \item \textbf{Multimodal Analysis:} Pedestrian and bicycle safety/delay
    \item \textbf{Network Effects:} Coordinated signal control across multiple intersections
    \item \textbf{Real-World Validation:} Calibration using field data from instrumented intersections
    \item \textbf{Connected/Autonomous Vehicles:} Impact of V2V and V2I communication
    \item \textbf{Environmental Impact:} Emissions and fuel consumption analysis
    \item \textbf{Safety Analysis:} Conflict point analysis and crash prediction
\end{enumerate}

\subsection{Final Remarks}

The choice between roundabout and signalized control is context-dependent. This research provides 
a quantitative framework for decision-making, but local factors (driver familiarity, land availability, 
pedestrian needs) must also be considered. Our optimization tools and visualizations enable rapid 
scenario evaluation to inform evidence-based design decisions.

\newpage

\section*{References}
\addcontentsline{toc}{section}{References}

\begin{enumerate}
    \item Webster, F. V. (1958). \textit{Traffic signal settings}. Road Research Technical Paper No. 39. 
          Road Research Laboratory, London.
    
    \item Treiber, M., Hennecke, A., \& Helbing, D. (2000). Congested traffic states in empirical observations 
          and microscopic simulations. \textit{Physical Review E}, 62(2), 1805.
    
    \item Krajzewicz, D., Erdmann, J., Behrisch, M., \& Bieker, L. (2012). Recent development and applications 
          of SUMO - Simulation of Urban MObility. \textit{International Journal On Advances in Systems and Measurements}, 
          5(3\&4), 128-138.
    
    \item Schulman, J., Wolski, F., Dhariwal, P., Radford, A., \& Klimov, O. (2017). Proximal policy optimization 
          algorithms. \textit{arXiv preprint arXiv:1707.06347}.
    
    \item Fernandes, P., Teixeira, J., Guarnaccia, C., Bandeira, J. M., Macedo, E., \& Coelho, M. C. (2020). 
          The potential of metering roundabouts: Influence on pollutant emissions, fuel consumption and traffic 
          flow. \textit{Transportation Research Part D: Transport and Environment}, 82, 102234.
    
    \item Akçelik, R. (1981). \textit{Traffic signals: Capacity and timing analysis}. Australian Road Research 
          Board Research Report ARR No. 123.
    
    \item Highway Capacity Manual (HCM) 2010. Transportation Research Board, National Research Council, 
          Washington, D.C.
    
    \item Rodegerdts, L., Blogg, M., Wemple, E., Myers, E., Kyte, M., Dixon, M., ... \& Carter, D. (2010). 
          \textit{Roundabouts: An informational guide} (Vol. 672). Transportation Research Board.
\end{enumerate}

\newpage

\appendix

\section{Simulation Parameters}

\subsection{Roundabout Parameters}

\begin{table}[H]
\centering
\caption{Roundabout Simulation Parameters}
\begin{tabular}{@{}lll@{}}
\toprule
\textbf{Parameter} & \textbf{Value} & \textbf{Units} \\ \midrule
Critical Gap & 4.0 & seconds \\
Gap Acceptance Std Dev & 0.5 & seconds \\
Max Speed & 50 & km/h \\
IDM Desired Speed & 13.89 & m/s \\
IDM Max Acceleration & 2.0 & m/s² \\
IDM Comfortable Deceleration & 3.0 & m/s² \\
IDM Minimum Spacing & 2.0 & meters \\
IDM Time Headway & 1.5 & seconds \\
Reaction Time & 1.0 & seconds \\
\bottomrule
\end{tabular}
\end{table}

\subsection{Signalized Intersection Parameters}

\begin{table}[H]
\centering
\caption{Signalized Intersection Simulation Parameters}
\begin{tabular}{@{}lll@{}}
\toprule
\textbf{Parameter} & \textbf{Value} & \textbf{Units} \\ \midrule
Minimum Green Time & 7 & seconds \\
Maximum Green Time & 60 & seconds \\
Yellow Time & 3 & seconds \\
All-Red Time & 2 & seconds \\
Startup Lost Time & 3 & seconds \\
Clearance Lost Time & 2 & seconds \\
Saturation Flow Rate & 1800 & veh/hr/lane \\
Minimum Cycle Length & 40 & seconds \\
Maximum Cycle Length & 180 & seconds \\
\bottomrule
\end{tabular}
\end{table}

\section{Code Availability}

All simulation code, configuration files, and analysis scripts are available at:

\texttt{https://github.com/username/intersection-optimization}

\section{Acknowledgments}

This research utilized the SUMO traffic simulation platform developed by the German Aerospace Center (DLR). 
Bayesian optimization employed the scikit-optimize library. Reinforcement learning implemented using 
Stable-Baselines3.

\end{document}
